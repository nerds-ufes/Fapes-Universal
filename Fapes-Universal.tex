\documentclass[10pt]{article}
\usepackage[english, portuges, brazil]{babel}    % hiphenação em portugues
%\usepackage[hypertex]{hyperref}
\usepackage{graphicx,url} % support the \includegraphics command and options

\usepackage{color,soul}

\usepackage[utf8]{inputenc}

\usepackage{hyperref}
\usepackage{url}
%   \def\UrlBreaks{} 
%   \def\UrlBigBreaks{\do\/\do-} 
%\usepackage{breakurl}
\usepackage{setspace}

% Muda o nome das tabelas para Quadro -- todas as seções são tabelas
\usepackage{caption}
\captionsetup[table]{name=Quadro}

%\usepackage[latin1]{inputenc} % set input encoding (not needed with XeLaTeX)
%\usepackage[latin1]{inputenc}
%\usepackage[num]{abntcite}
%\usepackage[alf]{abntcite}
\usepackage{todonotes}
%\usepackage[parfill]{parskip} % Activate to begin paragraphs with an empty line rather than an indent
%%% PACKAGES
\usepackage{booktabs} % for much better looking tables
\usepackage{array} % for better arrays (eg matrices) in maths
\usepackage{paralist} % very flexible & customisable lists (eg. enumerate/itemize, etc.)
\usepackage{verbatim} % adds environment for commenting out blocks of text & for better verbatim
\usepackage{subfigure} % make it possible to include more than one captioned figure/table in a single float
% These packages are all incorporated in the memoir class to one degree or another...

%%% PAGE DIMENSIONS
\usepackage{geometry} % to change the page dimensions
\geometry{a4paper,bottom=45mm,tmargin=10mm,lmargin=27mm,rmargin=20mm} 

%%% HEADERS & FOOTERS
\usepackage{fancyhdr} % This should be set AFTER setting up the page geometry
\pagestyle{fancy} % options: empty , plain , fancy
\setlength{\headheight}{80pt}
\renewcommand{\headrulewidth}{0pt}
\renewcommand{\footrulewidth}{0pt}
\lhead{}\chead{}
\rhead{\includegraphics[width=7cm]{figuras/logo_fapes.png}}
\lfoot{}
\cfoot{\small Fundação de Amparo à Pesquisa e Inovação do Espírito Santo \\
Tel.: 27 3636-1863/1864/1894 -- editais.duvidas@fapes.es.gov.br}
\rfoot{}

%%% SECTION TITLE APPEARANCE
\usepackage{sectsty}
\allsectionsfont{\sffamily\mdseries\upshape} % (See the fntguide.pdf for font help)
% (This matches ConTeXt defaults)

%%% ToC (table of contents) APPEARANCE
\usepackage[nottoc,notlof,notlot]{tocbibind} % Put the bibliography in the ToC
\usepackage[titles,subfigure]{tocloft} % Alter the style of the Table of Contents
\renewcommand{\cftsecfont}{\rmfamily\mdseries\upshape}
\renewcommand{\cftsecpagefont}{\rmfamily\mdseries\upshape} % No bold!

% tab size
\newcommand\tab[1][1cm]{\hspace*{#1}}

\usepackage{amssymb,amsmath}
\usepackage{longtable}

\usepackage{color, colortbl}
\definecolor{Gray}{gray}{0.9}

\usepackage{multirow}
%\usepackage{lscape}
\usepackage{pdflscape}

\usepackage[useregional=numeric]{datetime2}

%%%%%%%%%%%%%%%%%%%%%%%%%%%%%%%%%%%%%%%%%%%%%%%%%%%%%%%%%%%%%%%%%%%%%%%%%%%%%%%%%%%%%%%%%%%%%%%%%%%%%%%%%%%%%%%%%%%%%%%%%%%%
%%% Dados a preencher no formulário

% Nome da proposta
\newcommand{\propostaNome}{NOME COMPLETO DA PROPOSTA DE PESQUISA}

% Proponente
\newcommand{\proponenteNome}{NOME DO PROPONENTE}
\newcommand{\proponenteCPF}{000.000.000-00}
\newcommand{\proponenteRG}{0.000.000}
\newcommand{\proponenteRGEmissor}{SPTC/ES}
\newcommand{\proponenteEndereco}{ENDEREÇO RESIDENCIAL COMPLETO}
\newcommand{\proponenteBairro}{BAIRRO COMPLETO}
\newcommand{\proponenteCidade}{NOME DA CIDADE}
\newcommand{\proponenteUF}{ES}
\newcommand{\proponenteTelResidencial}{(27) 0000-0000}
\newcommand{\proponenteTelCelular}{(27) 0000-0000}
\newcommand{\proponenteTelComercial}{(27) 0000-0000}
\newcommand{\proponenteInstituicao}{NOME COMPLETO DA INSTITUIÇÃO -- SIGLA}
\newcommand{\proponenteDepartamento}{DEPARTAMENTO}
\newcommand{\proponenteCargo}{CARGO}
\newcommand{\proponenteEmails}{email1@provedor1.com.br; email2@provedor2.com.br}

% Instituição Executora
\newcommand{\instituicaoNome}{NOME COMPLETO DA INSTITUIÇÃO}
\newcommand{\instituicaoSigla}{SIGLA}
\newcommand{\instituicaoEndereco}{ENDEREÇO COMPLETO DA INSTITUIÇÃO}
\newcommand{\instituicaoCNPJ}{00.000.000/0000-00}
\newcommand{\instituicaoBairro}{BAIRRO COMPLETO}
\newcommand{\instituicaoMunicipio}{MUNICÍPIO}
\newcommand{\instituicaoUF}{ESPÍRITO SANTO}
\newcommand{\instituicaoRepresentante}{NOME DO REPRESENTANTE}
\newcommand{\instituicaoRepresentanteNacionalidade}{BRASILEIRO}
\newcommand{\instituicaoRepresentanteCargo}{REITOR}
\newcommand{\instituicaoRepresentanteCPF}{000.000.000-00}
\newcommand{\instituicaoRepresentanteRG}{0.000.000}
\newcommand{\instituicaoRepresentanteRGEmissor}{SPTC/ES}
\newcommand{\instituicaoRepresentanteTelefone}{(00) 0000-0000}
\newcommand{\instituicaoRepresentanteFax}{(00) 0000-0000}
\newcommand{\instituicaoRepresentanteEmail}{email1@instituicao.br}

\usepackage[none]{hyphenat}

\hypersetup
{
    pdfauthor={\proponenteNome},
    pdfsubject={Formulário FAPES de submissão -- Edital Universal},
    pdftitle={\propostaNome},
    pdfkeywords={}
}


\newcommand{\argmin}{\operatornamewithlimits{argmin}}
\newcommand{\divint}{\operatornamewithlimits{div}}

\newtheorem{theorem}{Theorem}
\newtheorem{lemma}[theorem]{Propriedade}
\sloppy



\begin{document}
%\maketitle
%\thispagestyle{empty}

\begin{center}
\begin{doublespace}
\textbf{ANEXO I \\
FORMULÁRIO FAPES DE SUBMISSÃO \\
EDITAL UNIVERSAL \\
}\end{doublespace}
\end{center}

%%%%%%%%%%%%%%%%%%%%%%%%%%%%%%%%%%%%%%%%%%%%%%%%%%%%%%%%%%%%%%%%%%%
% I. DADOS DO(A) PROPONENTE
\begin{longtable}{{|p{5.2cm}|p{5.2cm}|p{5.2cm}|}}
\hline
\multicolumn{3}{|l|}{\cellcolor[HTML]{C0C0C0}
\textbf{I. DADOS DO(A) PROPONENTE}} \\ 
\hline
\endfirsthead
%
\endhead
%
\multicolumn{3}{|l|}{\begin{tabular}[c]{@{}l@{}}
NOME:\\ 
\textbf{\proponenteNome}\end{tabular}} \\ \hline
\begin{tabular}[c]{@{}l@{}}
CPF:\\ 
\textbf{\proponenteCPF}
\end{tabular} & 
\begin{tabular}[c]{@{}l@{}}
CARTEIRA DE IDENTIDADE: \\ 
\textbf{\proponenteRG}
\end{tabular} & 
\begin{tabular}[c]{@{}l@{}}
ÓRGÃO:\\ 
\textbf{\proponenteRGEmissor}
\end{tabular} 
\\ \hline
\multicolumn{3}{|l|}{\begin{tabular}[c]{@{}l@{}}
ENDEREÇO RESIDENCIAL:\\ 
\textbf{\proponenteEndereco}
\end{tabular}} \\ 
\hline
\begin{tabular}[c]{@{}l@{}}
BAIRRO:\\ 
\textbf{\proponenteBairro}
\end{tabular} & 
\begin{tabular}[c]{@{}l@{}}
CIDADE:\\ 
\textbf{\proponenteCidade}
\end{tabular} & 
\begin{tabular}[c]{@{}l@{}}
ESTADO:\\ 
\textbf{\proponenteUF}
\end{tabular} \\ 
\hline
\begin{tabular}[c]{@{}l@{}}
TELEFONE RESIDENCIAL:\\ 
\textbf{\proponenteTelResidencial}
\end{tabular} & 
\begin{tabular}[c]{@{}l@{}}
TELEFONE CELULAR:\\ 
\textbf{\proponenteTelCelular}
\end{tabular} & 
\begin{tabular}[c]{@{}l@{}}
TELEFONE COMERCIAL:\\ 
\textbf{\proponenteTelComercial}
\end{tabular} \\ \hline
INSTITUIÇÃO DE VÍNCULO: & 
\multicolumn{2}{l|}{
\textbf{\proponenteInstituicao}} \\ 
\hline
DEPARTAMENTO: & 
\multicolumn{2}{l|}{
\textbf{\proponenteDepartamento}} \\ 
\hline
CARGO: & 
\multicolumn{2}{l|}{\textbf{\proponenteCargo}} \\ 
\hline
E-MAILS PESSOAIS: & 
\multicolumn{2}{l|}{\begin{tabular}[c]{@{}l@{}}
\textbf{\proponenteEmails}
\end{tabular}} \\ 
\hline
\end{longtable}


%%%%%%%%%%%%%%%%%%%%%%%%%%%%%%%%%%%%%%%%%%%%%%%%%%%%%%%%%%%%%%%%%%%
% II. DECLARAÇÃO DO(A) PROPONENTE
\begin{longtable}[c]{|l|c|l|}
\hline
\multicolumn{3}{|l|}{\cellcolor[HTML]{C0C0C0}
\textbf{II. DECLARAÇÃO DO(A) PROPONENTE}}\\ 
\hline
\multicolumn{3}{|l|}{\begin{tabular}[c]{@{}l@{}}
Declaro que: \\
\multicolumn{1}{m{16cm}}{
\begin{itemize}
    \item Tenho conhecimento da sistemática adotada pela FAPES para análise de solicitações neste Edital. 
    Autorizo que esta solicitação seja analisada segundo essa sistemática e, em particular, que ela seja submetida à análise de pesquisadores escolhidos pela FAPES, cujas identidades serão mantidas em sigilo.
    \item Tenho conhecimento de que é de minha total responsabilidade a obtenção de licenças e permissões junto aos órgãos pertinentes para realização da presente pesquisa.
    \item As informações aqui prestadas e as constantes em meu currículo para fins de submissão desta proposta foram por mim revisadas e estão corretas.
    \item Estou ciente de que as informações incorretas aqui prestadas poderão prejudicar a análise e eventual concessão desta solicitação.
\end{itemize}
} \\

\textbf{( X ) Declaração de concordância} \\[5pt]
\end{tabular}}  \\ 
\hline
\begin{tabular}[c]{@{}l@{}}Local\\ 
Vitória - ES
\end{tabular} & 
\begin{tabular}[c]{@{}l@{}}Data\\ 
\today
\end{tabular} & 
\begin{tabular}[c]{@{}l@{}}Assinatura do(a) proponente \\ 
{\color{red}(Não é necessário para submissão via SigFapes)}
\end{tabular} \\ 
\hline
\end{longtable}


%%%%%%%%%%%%%%%%%%%%%%%%%%%%%%%%%%%%%%%%%%%%%%%%%%%%%%%%%%%%%%%%%%%
% III. DADOS DA INSTITUIÇÃO EXECUTORA
\begin{longtable}[c]{|p{5.2cm}|p{5.2cm}|p{5.2cm}|}
\hline
\multicolumn{3}{|l|}{\cellcolor[HTML]{C0C0C0}
\textbf{III. DADOS DA INSTITUIÇÃO EXECUTORA}} \\ 
\hline
\multicolumn{2}{|l|}{\begin{tabular}[c]{@{}l@{}}NOME DA INSTITUIÇÃO:\\ 
\textbf{\instituicaoNome}
\end{tabular}} & 
\begin{tabular}[c]{@{}l@{}}SIGLA:\\ 
\textbf{\instituicaoSigla}
\end{tabular} \\
\hline
\multicolumn{2}{|l|}{\begin{tabular}[c]{@{}l@{}}ENDEREÇO:\\ 
\textbf{\instituicaoEndereco}
\end{tabular}} & 
\begin{tabular}[c]{@{}l@{}}CNPJ:\\
\textbf{\instituicaoCNPJ}
\end{tabular} \\ 
\hline
\begin{tabular}[c]{@{}l@{}}BAIRRO:\\ 
\textbf{\instituicaoBairro}
\end{tabular} & 
\begin{tabular}[c]{@{}l@{}}MUNICÍPIO:\\ 
\textbf{\instituicaoMunicipio}
\end{tabular} & 
\begin{tabular}[c]{@{}l@{}}ESTADO:\\ 
\textbf{\instituicaoUF}
\end{tabular}   \\ 
\hline
\multicolumn{3}{|l|}{\begin{tabular}[c]{@{}l@{}}NOME DO REPRESENTANTE LEGAL ou REPRESENTANTE POR DELEGAÇÃO:\\ 
\textbf{\instituicaoRepresentante}
\end{tabular}} \\ 
\hline
\begin{tabular}[c]{@{}l@{}}NACIONALIDADE:\\
\textbf{\instituicaoRepresentanteNacionalidade}
\end{tabular} & 
\multicolumn{2}{l|}{\begin{tabular}[c]{@{}l@{}}CARGO E ATO DE NOMEAÇÃO/DELEGAÇÃO:\\ 
\textbf{\instituicaoRepresentanteCargo}
\end{tabular}} \\ 
\hline
\begin{tabular}[c]{@{}l@{}}CPF:\\ 
\textbf{\instituicaoRepresentanteCPF}
\end{tabular} & 
\begin{tabular}[c]{@{}l@{}}CARTEIRA IDENTIDADE:\\ 
\textbf{\instituicaoRepresentanteRG}
\end{tabular} & 
\begin{tabular}[c]{@{}l@{}}ÓRGÃO:\\ 
\textbf{\instituicaoRepresentanteRGEmissor}
\end{tabular} \\ 
\hline
\begin{tabular}[c]{@{}l@{}}TELEFONE:\\ 
\textbf{\instituicaoRepresentanteTelefone}
\end{tabular} & 
\begin{tabular}[c]{@{}l@{}}FAX:\\ 
\textbf{\instituicaoRepresentanteFax}
\end{tabular} & 
\begin{tabular}[c]{@{}l@{}}E-MAIL: \\
\textbf{\instituicaoRepresentanteEmail}
\end{tabular}  \\ 
\hline
\end{longtable}


%%%%%%%%%%%%%%%%%%%%%%%%%%%%%%%%%%%%%%%%%%%%%%%%%%%%%%%%%%%%%%%%%%%
% IV. TERMOS DE COMPROMISSO E CONCORDÂNCIA DA INSTITUIÇÃO EXECUTORA
\begin{longtable}[c]{|p{3cm}|p{4cm}|p{6cm}|}
\hline
\multicolumn{3}{|l|}{\cellcolor[HTML]{C0C0C0}
\small \textbf{IV. TERMOS DE COMPROMISSO E CONCORDÂNCIA DA INSTITUIÇÃO  EXECUTORA}} \\ 
\hline
\multicolumn{3}{|l|}{\begin{tabular}[c]{@{}l@{}}
Declaro que: \\
% delimita o tamanho do texto, para não estourar a tabela
\multicolumn{1}{m{16cm}}{
Declaro que estou ciente das necessidades infraestruturais demandadas para a execução do projeto (\textbf{\propostaNome}), submetido ao presente edital. Declaro ainda que o(a) pesquisador(a) proponente \proponenteNome vinculado a esta Instituição terá todo apoio institucional necessário para a realização do referido projeto, com garantia do espaço físico, instalações (laboratórios, rede de computação, base de dados, etc.), assegurando a contrapartida de recursos materiais e humanos, bem como o acesso a todos os serviços disponíveis na Instituição e relevantes para sua execução.
} \\
\end{tabular}}  \\ 
\hline
\begin{tabular}[c]{@{}l@{}}Data\\ 
\textbf{\today}
\end{tabular} & 
\begin{tabular}[c]{@{}l@{}}Cargo/função \\ 
\textbf{\instituicaoRepresentanteCargo}
\end{tabular} & 
\begin{tabular}[c]{@{}l@{}}Assinatura/Carimbo do representante legal ou \\
representante por delegação: \\
{\color{red}(Não é necessário para submissão via SigFapes)}\end{tabular} \\ 
\hline
\end{longtable}

%%%%%%%%%%%%%%%%%%%%%%%%%%%%%%%%%%%%%%%%%%%%%%%%%%%%%%%%%%%%%
% V. DADOS DAS DEMAIS INSTITUIÇÕES DOS PESQUISADORES DOUTORES
\begin{longtable}[c]{|p{8.7cm}|l|l|l|}
\hline
\multicolumn{4}{|l|}{\cellcolor[HTML]{C0C0C0}\begin{tabular}[c]{@{}l@{}}
\textbf{V. DADOS DAS DEMAIS INSTITUIÇÕES DOS PESQUISADORES DOUTORES}\\ 
\textbf{{\small\color{red}(PESQUISADORES PRINCIPAIS)}}\end{tabular}}\\ 
\hline
\endfirsthead
%
\endhead
%

% Instituição 1
\multicolumn{2}{|l|}{\begin{tabular}[c]{@{}l@{}}NOME DA INSTITUIÇÃO:\\ 
\textbf{Nome da Instituição principal 1}
\end{tabular}} & 
\multicolumn{2}{l|}{\begin{tabular}[c]{@{}l@{}}SIGLA:\\ 
\textbf{SIGLA}
\end{tabular}}\\ 
\hline
NOME E CARGO DO REPRESENTANTE LEGAL: & 
\multicolumn{3}{l|}{\begin{tabular}[c]{@{}l@{}}
\textbf{Nome do Representante Legal} \\ 
\textbf{(Cargo)}
\end{tabular}} \\ 
\hline
\multicolumn{4}{|l|}{
\begin{tabular}[c]{@{}l@{}}ENDEREÇO:\\ 
\textbf{Endereço linha 1} \\
\textbf{Endereço linha 2} \\
\textbf{Endereço linha 3} 
\end{tabular}}\\ 
\hline
\begin{tabular}[c]{@{}l@{}}BAIRRO:\\ 
\textbf{Bairro}
\end{tabular}   & 
\begin{tabular}[c]{@{}l@{}}CIDADE:\\ 
\textbf{Cidade}
\end{tabular} & 
\begin{tabular}[c]{@{}l@{}}CEP:\\ 
\textbf{00000-000}
\end{tabular} & 
\begin{tabular}[c]{@{}l@{}}ESTADO:\\ 
\textbf{UF}
\end{tabular} \\ 
\hline
\begin{tabular}[c]{@{}l@{}}TELEFONE:\\ 
\textbf{(00) 0000-0000}
\end{tabular} & 
\multicolumn{3}{l|}{\begin{tabular}[c]{@{}l@{}}EMAIL:\\ 
\textbf{email@instituicao.br}
\end{tabular}}\\ 
\hline

% Instituição 2
\multicolumn{2}{|l|}{\begin{tabular}[c]{@{}l@{}}NOME DA INSTITUIÇÃO:\\ 
\textbf{Nome da Instituição principal 2}
\end{tabular}} & 
\multicolumn{2}{l|}{\begin{tabular}[c]{@{}l@{}}SIGLA:\\ 
\textbf{SIGLA}
\end{tabular}}\\ 
\hline
NOME E CARGO DO REPRESENTANTE LEGAL: & 
\multicolumn{3}{l|}{\begin{tabular}[c]{@{}l@{}}
\textbf{Nome do Representante Legal} \\ 
\textbf{(Cargo)}
\end{tabular}} \\ 
\hline
\multicolumn{4}{|l|}{
\begin{tabular}[c]{@{}l@{}}ENDEREÇO:\\ 
\textbf{Endereço linha 1} \\
\textbf{Endereço linha 2} \\
\textbf{Endereço linha 3} 
\end{tabular}}\\ 
\hline
\begin{tabular}[c]{@{}l@{}}BAIRRO:\\ 
\textbf{Bairro}
\end{tabular}   & 
\begin{tabular}[c]{@{}l@{}}CIDADE:\\ 
\textbf{Cidade}
\end{tabular} & 
\begin{tabular}[c]{@{}l@{}}CEP:\\ 
\textbf{00000-000}
\end{tabular} & 
\begin{tabular}[c]{@{}l@{}}ESTADO:\\ 
\textbf{UF}
\end{tabular} \\ 
\hline
\begin{tabular}[c]{@{}l@{}}TELEFONE:\\ 
\textbf{(00) 0000-0000}
\end{tabular} & 
\multicolumn{3}{l|}{\begin{tabular}[c]{@{}l@{}}EMAIL:\\ 
\textbf{email@instituicao.br}
\end{tabular}}\\ 
\hline

\end{longtable}

%%%%%%%%%%%%%%%%%%%%%%%%%%%%%%%%%%%%%%%%%%%%%%%%%%%%%%%%%%%%%%%%%%%%%%%%%%%%%%%%%%%%%%%%%%%%%%%%%%%%%%%%%%%%%%%%%%%%%%%%%%
% VI. EQUIPE EXECUTORA DO PROJETO (item 8.3 do edital)
\begin{longtable}{|c|c|c|}
\hline
\multicolumn{3}{|l|}{\cellcolor[HTML]{C0C0C0}\textbf{\begin{tabular}[c]{@{}l@{}}
VI. EQUIPE EXECUTORA DO PROJETO \\ 
{\small\color{red}(item 8.3 do edital) -- inserir quantas linhas forem necessárias}\end{tabular}}} \\ 
\hline
\endfirsthead
%
\endhead
%
\multicolumn{3}{|c|}{\cellcolor[HTML]{C0C0C0}\textbf{PESQUISADORES PRINCIPAIS*}} \\ \hline
\textbf{Nome do(a) pesquisador(a)} & 
\textbf{\begin{tabular}[c]{@{}c@{}}Titulação\\ máxima\end{tabular}} & 
\textbf{\begin{tabular}[c]{@{}c@{}}Instituição/Departamento/\\ Laboratório\end{tabular}} \\ \hline
\proponenteNome & 
Doutorado & 
\begin{tabular}[c]{@{}c@{}}
\proponenteInstituicao /\\ 
Departamento de \proponenteDepartamento / \\ 
Laboratório proponente (SIGLA)
\end{tabular} \\ 
\hline

Pesquisador Principal 1 & 
Doutorado & 
\begin{tabular}[c]{@{}c@{}}
\proponenteInstituicao / \\ 
Departamento pesquisador principal 1 / \\ 
Laboratório pesquisador principal 1
\end{tabular} \\ 
\hline

Pesquisador Principal 2 & 
Doutorado & 
\begin{tabular}[c]{@{}c@{}}
\proponenteInstituicao / \\ 
Departamento pesquisador principal 2 / \\ 
Laboratório pesquisador principal 2
\end{tabular} \\ 
\hline

Pesquisador Principal 3 & 
Doutorado & 
\begin{tabular}[c]{@{}c@{}}
\proponenteInstituicao / \\ 
Departamento pesquisador principal 3 / \\ 
Laboratório pesquisador principal 3
\end{tabular} \\ 
\hline

\end{longtable}
\vspace*{-14pt}
\noindent{\color{red}*Indicar somente os pesquisadores principais, conforme item 8.3.a-c, dos quais deverá ser apresentada cópia do currículo Lattes (para faixas B e C).}



%%%%%%%%%%%%%%%%%%%%%%%%%%%%%%%%%%%%%%%%%%%%%%%%%%%%%%%%%%%%%%%%%%%%%%%%%%%%%%%%%%%%%%%%%%%%%%%%%%%%%%%%%%%%%%%%%%%%%%%%%%
% Pesquisadores Colaboradores
\begin{longtable}{|c|c|c|}
\hline
\multicolumn{3}{|c|}{\cellcolor[HTML]{C0C0C0}
\textbf{PESQUISADORES COLABORADORES**}} \\ 
\hline
\endfirsthead
%
\endhead
%
\textbf{Nome do(a) pesquisador(a)} & \textbf{Titulação máxima} & \textbf{\begin{tabular}[c]{@{}c@{}}Instituição/Departamento/\\ Laboratório\end{tabular}} \\ 
\hline

Pesquisador colaborador 1 & 
Titulação &
\begin{tabular}[c]{@{}c@{}}
Instituição colaborador 1 /\\ 
Departamento /\\
Laboratório
\end{tabular} \\ 
\hline

Pesquisador colaborador 2 & 
Titulação &
\begin{tabular}[c]{@{}c@{}}
Instituição colaborador 2 /\\ 
Departamento /\\
Laboratório
\end{tabular} \\ 
\hline

Pesquisador colaborador 3 & 
Titulação &
\begin{tabular}[c]{@{}c@{}}
Instituição colaborador 3 /\\ 
Departamento /\\
Laboratório
\end{tabular} \\ 
\hline

Pesquisador colaborador 4 & 
Titulação &
\begin{tabular}[c]{@{}c@{}}
Instituição colaborador 4 /\\ 
Departamento /\\
Laboratório
\end{tabular} \\ 
\hline

Pesquisador colaborador 5 & 
Titulação &
\begin{tabular}[c]{@{}c@{}}
Instituição colaborador 5 /\\ 
Departamento /\\
Laboratório
\end{tabular} \\ 
\hline

\end{longtable}
\vspace*{-14pt}
\noindent{\color{red}**São os demais pesquisadores colaboradores. Não necessita entrega do currículo Lattes.}



%%%%%%%%%%%%%%%%%%%%%%%%%%%%%%%%%%%%%%%%%%%%%%%%%%%%%%%%%%%%%%%%%%%%%%%%%%%%%%%%%%%%%%%%%%%%%%%%%%%%%%%%%%%%%%%%%%%%%%%%%%
% Demais membros
\begin{longtable}{|c|c|c|}
\hline
\multicolumn{3}{|c|}{\cellcolor[HTML]{C0C0C0}\textbf{DEMAIS MEMBROS***}} \\ 
\hline
\endfirsthead
%
\endhead
%
\textbf{Nome do(a) pesquisador(a)} & \textbf{Titulação} & \textbf{Instituição/PPG/Laboratório} \\ 
\hline

Membro/Aluno 1 & 
Titulação &
\begin{tabular}[c]{@{}c@{}}
Instituição /\\ 
PPG /\\
Laboratório
\end{tabular} \\ 
\hline

Membro/Aluno 2 & 
Titulação &
\begin{tabular}[c]{@{}c@{}}
Instituição /\\ 
PPG /\\
Laboratório
\end{tabular} \\ 
\hline

Membro/Aluno 3 & 
Titulação &
\begin{tabular}[c]{@{}c@{}}
Instituição /\\ 
PPG /\\
Laboratório
\end{tabular} \\ 
\hline

Membro/Aluno 4 & 
Titulação &
\begin{tabular}[c]{@{}c@{}}
Instituição /\\ 
PPG /\\
Laboratório
\end{tabular} \\ 
\hline

Membro/Aluno 5 & 
Titulação &
\begin{tabular}[c]{@{}c@{}}
Instituição /\\ 
PPG /\\
Laboratório
\end{tabular} \\ 
\hline

\end{longtable}
\vspace*{-14pt}
\noindent{\color{red}***São os demais membros do projeto, p.ex. alunos. Não necessita entrega do currículo Lattes.}

%\begin{small}
\begin{longtable}[c]{|l|l|l|}
\hline
\multicolumn{3}{|m{16.6cm}|}{\cellcolor[HTML]{C0C0C0}
\textbf{VII. IDENTIFICAÇÃO DA PROPOSTA DE PESQUISA}} \\ 
\hline
\endfirsthead
%
\endhead
%
\multicolumn{3}{|l|}{
\textbf{\begin{tabular}[c]{@{}l@{}}
Título do Projeto: \\ 
\propostaNome
\end{tabular}}}\\ 
\hline
\multicolumn{3}{|l|}{\textbf{Faixa:} (~~~) A~~~~~(~~~) B~~~~~(~X~) C} \\
\hline
\textbf{\begin{tabular}[c]{@{}l@{}}Tipo	de\\ Pesquisa:\end{tabular}} & 
\begin{tabular}[c]{@{}l@{}}
(~~~) Pesquisa Básica\tab(~X~) Pesquisa Aplicada   \\ 
(~~~) Desenvolvimento\tab(~~~) Transferência de Tecnologia
\end{tabular} & 
\begin{tabular}[c]{@{}l@{}}
(~X~) Experimental \\ 
(~~~) Não-Experimental
\end{tabular} \\ 
\hline
\end{longtable}
%\end{small}


%\begin{small}
\begin{longtable}[c]{|l|l|l|l|}
\hline
\multicolumn{4}{|m{16.6cm}|}{\cellcolor[HTML]{C0C0C0}
\textbf{Descrição da(s) localidade/Município(s) onde a Pesquisa será realizada}} \\ 
\hline
\endfirsthead
%
\endhead
%
\multicolumn{4}{|l|}{
\textbf{\begin{tabular}[c]{@{}l@{}}
Município - UF \\
Município - UF \\
Município - UF \\
Município - UF \\
Município - UF \\
\end{tabular}}} \\
\hline
\multicolumn{4}{|l|}{\cellcolor[HTML]{C0C0C0}\textbf{Grande Área do Conhecimento da Proposta:}} \\ \hline
~~ & Ciências Agrárias                       & ~~ & Linguística, Letras e Artes              \\ \hline
~~ & Engenharias                             & ~~ & Ciências da Saúde                        \\ \hline
X  & Ciências Exatas e da Terra              & ~~ & Ciências Sociais Aplicadas               \\ \hline
~~ & Ciências Humanas                        & ~~ & Ciências da Vida                         \\ \hline
\multicolumn{4}{|l|}{\cellcolor[HTML]{C0C0C0}\textbf{Subáreas do Conhecimento da Proposta (conforme tabela do CNPq)}} \\ \hline
\multicolumn{4}{|l|}{1.03.04.04-5 Teleinformática}                                                                   \\ \hline
\multicolumn{4}{|l|}{1.03.04.02-9 Arquitetura de Sistemas de Computação}                                             \\ \hline
\multicolumn{4}{|l|}{1.03.04.03-7 Software Básico}                                                                   \\ \hline
\end{longtable}

\begin{longtable}[c]{l}
\hline
\rowcolor[HTML]{C0C0C0} 
\multicolumn{1}{|m{16.6cm}|}{\cellcolor[HTML]{C0C0C0}
\textbf{VIII. RESUMO DO PROJETO}} \\ 
\hline
\endfirsthead
%
\endhead
%
\end{longtable}

Texto do resumo.
\\

\noindent\underline{Palavras-chave}: Palavra-chave 1, Palavra-chave 2, Palavra-chave 3

\begin{longtable}[c]{l}
\hline
\rowcolor[HTML]{C0C0C0} 
\multicolumn{1}{|m{16.6cm}|}{\cellcolor[HTML]{C0C0C0}
\textbf{IX. CARACTERIZAÇÃO DO PROBLEMA CIENTÍFICO E/OU TECNOLÓGICO A SER ABORDADO}} \\ 
\hline
\endfirsthead
%
\endhead
%
\end{longtable}

Texto da caracterização do problema científico e/ou tecnológico a ser abordado.

\begin{longtable}[c]{l}
\hline
\rowcolor[HTML]{C0C0C0} 
\multicolumn{1}{|m{16.6cm}|}{\cellcolor[HTML]{C0C0C0}
\textbf{X. OBJETIVO GERAL}} \\ 
\hline
\endfirsthead
%
\endhead
%
\end{longtable}

Texto do objetivo geral.

\begin{longtable}[c]{l}
\hline
\rowcolor[HTML]{C0C0C0}
\multicolumn{1}{|m{16.6cm}|}{\cellcolor[HTML]{C0C0C0}
\textbf{XI. OBJETIVOS ESPECÍFICOS/METAS}} \\ 
\hline
\endfirsthead
%
\endhead
%
\end{longtable}

Texto dos objetivos específicos.

\begin{longtable}[c]{l}
\hline
\rowcolor[HTML]{C0C0C0} 
\multicolumn{1}{|m{16.6cm}|}{\cellcolor[HTML]{C0C0C0}
\textbf{XII. METODOLOGIA}} \\ 
\hline
\endfirsthead
%
\endhead
%
\end{longtable}

Texto da metodologia a ser utilizada.

\begin{longtable}[c]{l}
\hline
\rowcolor[HTML]{C0C0C0} 
\multicolumn{1}{|m{16.6cm}|}{\cellcolor[HTML]{C0C0C0}
\textbf{XIII. RESULTADOS ESPERADOS E IMPACTOS NA SOCIEDADE}} \\ 
\hline
\endfirsthead
%
\endhead
%
\end{longtable}

Texto dos resultados esperados e impactos na sociedade.

\begin{longtable}[c]{l}
\hline
\rowcolor[HTML]{C0C0C0} 
\multicolumn{1}{|l|}{\cellcolor[HTML]{C0C0C0}\textbf{\begin{tabular}[c]{@{}l@{}}
XIV. EXPERIÊNCIA E QUALIFICAÇÃO DO(A) COORDENADOR(A) EM RELAÇÃO \\
A FORMAÇÃO DE RECURSOS HUMANOS, PRODUÇÃO TÉCNICO-CIENTÍFICA \\
E/OU DE INOVAÇÃO E COORDENAÇÃO DE PROJETOS\end{tabular}}} \\ 
\hline
\endfirsthead
%
\endhead
%
\end{longtable}

Texto da experiência e qualificação do(a) coordenador(a).

% Dica: utilize o site [https://www.tablesgenerator.com/] para criar a tabela do plano de metas em Latex.

% Página em modo paisagem para acomodar a planilha de plano de metas e indicadores de progresso
\begin{landscape}
\begin{small}

\begin{longtable}[c]{|c|c|c|c|c|c|}
\hline
\multicolumn{6}{|l|}{\cellcolor[HTML]{C0C0C0}
\textbf{XV. Plano de metas e indicadores de progressos}} \\ \hline
%
 &  & \multicolumn{4}{c|}{\textbf{Cronograma de execução}} \\ \cline{3-6} 
\multirow{-2}{*}{\textbf{Meta}} & \multirow{-2}{*}{\textbf{Atividade}} & \textbf{Semestre 1} & \textbf{Semestre 2} & \textbf{Semestre 3} & \textbf{Semestre 4} \\ \hline
\endhead

 & \begin{tabular}[c]{@{}c@{}}Descrição da\\ Atividade 1.1\end{tabular} & Desc A & Desc B & - & - \\ \cline{2-6} 
\multirow{-2}{*}{Descrição da Meta 1} & \begin{tabular}[c]{@{}c@{}}Descrição da\\ Atividade 1.2\end{tabular} & - & Desc C & Desc D & - \\ \hline
 & \begin{tabular}[c]{@{}c@{}}Descrição da \\ Atividade 2.1\end{tabular} & - & Desc E & - & - \\ \cline{2-6} 
\multirow{-2}{*}{Descrição da Meta 2} & \begin{tabular}[c]{@{}c@{}}Descrição da \\ Atividade 2.2\end{tabular} & - & - & Desc F & - \\ \hline
 & - & - & - & - & - \\ \cline{2-6} 
\multirow{-2}{*}{-} & - & - & - & - & - \\ \hline
 & - & - & - & - & - \\ \cline{2-6} 
\multirow{-2}{*}{-} & - & - & - & - & - \\ \hline
- & - & - & - & - & - \\ \cline{2-6} 
 & - & - & - & - & - \\ \hline
\end{longtable}

\end{small}
\end{landscape}

\newpage

\begin{longtable}[c]{|c|c|c|c|c|c|c|c|c|c|}
\hline
\multicolumn{10}{|l|}{\cellcolor[HTML]{C0C0C0}
\textbf{XVI. CRONOGRAMA {\small\color{red}(Marcar com um X)}}} \\ \hline
\endfirsthead
%
\endhead
%
 &  & \multicolumn{8}{c|}{\textbf{Trimestres}} \\ \cline{3-10} 
\multirow{-2}{*}{\textbf{\begin{tabular}[c]{@{}c@{}}Nº da\\ Meta\end{tabular}}} & \multirow{-2}{*}{\textbf{\begin{tabular}[c]{@{}c@{}}Nº da\\ Atividade\end{tabular}}} & 
\textbf{~~~1~~~} & \textbf{~~~2~~~} & \textbf{~~~3~~~} & \textbf{~~~4~~~} & \textbf{~~~5~~~} & \textbf{~~~6~~~} & \textbf{~~~7~~~} & \textbf{~~~8~~~} \\ \hline
 & 1.1 & ~ & ~ & ~ & ~ & ~ & ~ & ~ &  \\ \cline{2-10} 
\multirow{-2}{*}{Meta 1} & 1.2 & ~ & ~ & ~ & ~ & ~ & ~ & ~ &  \\ \hline
 & 2.1 & ~ & ~ & ~ & ~ & ~ & ~ & ~ &  \\ \cline{2-10} 
\multirow{-2}{*}{Meta 2} & 2.2 &  & ~ & ~ & ~ & ~ & ~ & ~ &  \\ \hline
 & 3.1 & ~ & ~ & ~ & ~ & ~ & ~ & ~ &  \\ \cline{2-10} 
\multirow{-2}{*}{Meta 3} & 3.2 & ~ & ~ & ~ & ~ & ~ & ~ & ~ &  \\ \hline
 & 4.1 & ~ & ~ & ~ & ~ & ~ & ~ & ~ &  \\ \cline{2-10} 
\multirow{-2}{*}{Meta 4} & 4.2 & ~ & ~ & ~ & ~ & ~ & ~ & ~ &  \\ \hline
 & 5.1 & ~ & ~ & ~ & ~ & ~ & ~ & ~ &  \\ \cline{2-10} 
\multirow{-2}{*}{Meta 5} & 5.2 & ~ & ~ & ~ & ~ & ~ & ~ & ~ &  \\ \hline
 & 6.1 &  & ~ & ~ & ~ & ~ & ~ & ~ &  \\ \cline{2-10} 
\multirow{-2}{*}{Meta 6} & 6.2 & ~ & ~ & ~ & ~ & ~ & ~ & ~ &  \\ \hline
Meta 7 & 7.1 & ~ & ~ & ~ & ~ & ~ & ~ & ~ & ~ \\ \hline
Meta 8 & 8.1 & ~ & ~ & ~ & ~ & ~ & ~ & ~ & ~ \\ \hline
Meta 9 & 9.1 & ~ & ~ & ~ & ~ & ~ & ~ & ~ & ~ \\ \hline
\end{longtable}

\begin{longtable}[c]{l}
\hline
\rowcolor[HTML]{C0C0C0} 
\multicolumn{1}{|m{16.6cm}|}{\cellcolor[HTML]{C0C0C0}\textbf{\begin{tabular}[c]{@{}l@{}}
XVII. RISCOS E DIFICULDADES NO DESENVOLVIMENTO DA PESQUISA
\end{tabular}}} \\ 
\hline
\endfirsthead
%
\endhead
%
\end{longtable}

Texto sobre riscos e dificuldades.

\begin{longtable}[c]{l}
\hline
\rowcolor[HTML]{C0C0C0} 
\multicolumn{1}{|m{16.6cm}|}{\cellcolor[HTML]{C0C0C0}\textbf{\begin{tabular}[c]{@{}l@{}}
XVIII. IMPORTÂNCIA DA PESQUISA PARA O DESENVOLVIMENTO DE \\
INSTITUIÇÕES DE FORA DA REGIÃO METROPOLITANA
\end{tabular}}} \\ \hline
\endfirsthead
%
\endhead
%
\end{longtable}

Texto sobre a importância da pesquisa para o desenvolvimento de instituições de fora da região metropolitana.

\begin{longtable}[c]{l}
\hline
\rowcolor[HTML]{C0C0C0} 
\multicolumn{1}{|m{16.6cm}|}{\cellcolor[HTML]{C0C0C0}\textbf{\begin{tabular}[c]{@{}l@{}}
XIX. POTENCIAL DA PESQUISA NA FORMAÇÃO DE RECURSOS HUMANOS PARA O \\ ESTADO DO ESPÍRITO SANTO
\end{tabular}}} \\ \hline
\endfirsthead
%
\endhead
%
\end{longtable}

Texto sobre o potencial da pesquisa na formação de recursos humanos para o ES.

\begin{longtable}[c]{l}
\hline
\rowcolor[HTML]{C0C0C0} 
\multicolumn{1}{|m{16.6cm}|}{\cellcolor[HTML]{C0C0C0}\textbf{\begin{tabular}[c]{@{}l@{}}
XX. POTENCIAL DA PROPOSTA PARA O FORTALECIMENTO DE LINHAS DE \\
PESQUISA E NUCLEAÇÃO DE NOVAS LINHAS DE PESQUISA
\end{tabular}}} \\ \hline
\endfirsthead
%
\endhead
%
\end{longtable}

Texto sobre o potencial da proposta para o fortalecimento de linhas de pesquisa e nucleação de novas linhas.

\bibliographystyle{plain}

\bibliography{Fapes-Universal}

\end{document}